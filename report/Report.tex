\documentclass[a4paper,12pt]{article}
\usepackage[utf8]{inputenc}
\usepackage[margin=2.5cm]{geometry}
\usepackage{graphicx}
\usepackage{xcolor}
\usepackage{amsmath}
\usepackage{hyperref}
\usepackage{tikz}
\usepackage{booktabs} % For professional tables
\usepackage{multirow} % For spanning rows
\usepackage{colortbl} % For colored tables
\usepackage{xcolor}

% Define a light blue color for table header
\definecolor{lightblue}{RGB}{173, 216, 230}

\hypersetup{
    colorlinks=true,
    linkcolor=blue,
    filecolor=magenta,
    urlcolor=blue,
}

\begin{document}

\pagenumbering{gobble}

\begin{center}            
    {\LARGE\textbf{COS314 Assignment Three}}\\
    
    \vspace{4cm}
    
    {\Large\textbf{Aplication of Machine Learning Alogorithms to Stock Trading Prediction}}\\
    \vspace{0.5cm}
    {\large\textbf{Using GP, MLP and Decision Trees}}\\
    
    \vspace{12cm}

    {\large Team:}\\
    \vspace{0.3cm}
    {\large Tinotenda Chirozvi (22547747)}\\
    {\large Lubabalo Tshikila (22644106)}\\
    {\large Dean Ramsay (22599012)}\\
    
    \vspace{2cm}
    
    {\large\textbf{Due Date: 24 May 2025}}\\
    
\end{center}

\clearpage
\pagenumbering{arabic} % Start with page number 1

\section*{Abstract}
This report presents the design, implementation, and evaluation of three machine learning algorithms for the classification of stock market movements. In particular the movement of the bitcoin value to predict whether it will rise(buy) or fall (sell). The three algorithms used: Genetic Programming Algorithm (GP), Decision Tree Algorithm (DT) and a Multi-Layer-Perceptron Algorithm (MLP).

\section{Introduction}
Data of historic BTC stock movements were provided, already divided into training and test data as BTC\_train.csv and BTC\_test.csv respectively.
The training data was used to train the three models which were then evaluated using the test data to verify their performance.


\section{Machine Learning Algorithms}

\subsection{Genetic Programming Algorithm}
\subsubsection{Algorithm Overview}

\paragraph{Problem Formulation}
The GP classifier addresses a binary classification problem where input features represent historical financial indicators, and the output determines whether a stock should be purchased (class 1) or not (class 0). The algorithm evolves mathematical expressions that map five normalized financial features to a classification decision through sigmoid activation.

\subsubsection{Tree Representation Structure}
\paragraph{Node Architecture}
The GP implementation employs a heterogeneous tree structure comprising terminal and non-terminal nodes. The Node interface defines the fundamental operations required for tree evaluation and manipulation:

\textbf{Terminal Nodes:}
\begin{itemize}
    \item \textbf{FeatureNode}: Represents input variables ($x_0$ through $x_4$) corresponding to the five financial features.
    \item \textbf{ConstantNode}: Contains randomly generated constant values in the range $[-1, 1]$ to provide numerical flexibility.
\end{itemize}

\textbf{Non-Terminal Nodes:}
\begin{itemize}
    \item \textbf{AddNode}: Performs addition operations between child nodes.
    \item \textbf{SubtractNode}: Executes subtraction operations between child nodes.
    \item \textbf{MultiplyNode}: Implements multiplication operations between child nodes.
    \item \textbf{SafeDivideNode}: Conducts protected division with safeguards against division by zero.
\end{itemize}

\paragraph{Tree Generation Strategy}
The algorithm implements a ramped half-and-half initialization strategy, alternating between grow and full methods during tree construction. Trees are constrained to a maximum depth of 4 levels and a maximum node count of 100 to prevent excessive bloat while maintaining expression diversity.

\subsubsection{Genetic Operators Implementation}
\paragraph{Crossover Operation}
The crossover operator implements subtree exchange between two parent individuals. The process involves:
\begin{enumerate}
    \item \textbf{Subtree Selection}: Random selection of subtrees from both parent trees using uniform distribution.
    \item \textbf{Subtree Exchange}: Replacement of the selected subtree in the first parent with the corresponding subtree from the second parent.
    \item \textbf{Offspring Generation}: Creation of a new individual containing the modified tree structure.
\end{enumerate}
This approach maintains syntactic correctness while exploring new combinations of existing genetic material.

\paragraph{Mutation Operation}
Mutation introduces genetic diversity through subtree replacement. The implementation:
\begin{enumerate}
    \item \textbf{Target Selection}: Random identification of a subtree within the individual.
    \item \textbf{Subtree Generation}: Creation of a new random subtree using the same construction method as initialization.
    \item \textbf{Replacement}: Substitution of the selected subtree with the newly generated structure.
\end{enumerate}
The mutation rate is set to 15\% to balance exploration and exploitation during evolution.

\paragraph{Selection Mechanism}
Tournament selection serves as the parent selection method with a tournament size of 5 individuals. This approach:
\begin{enumerate}
    \item \textbf{Candidate Sampling}: Random selection of 5 individuals from the current population.
    \item \textbf{Fitness Comparison}: Identification of the individual with highest fitness value.
    \item \textbf{Parent Selection}: Return of the best individual as a parent for reproduction.
\end{enumerate}
Tournament selection maintains selection pressure while preserving population diversity compared to purely elitist approaches.

\subsubsection{Fitness Evaluation Framework}
\paragraph{Classification Process}
The fitness evaluation process transforms continuous tree outputs into binary classifications:
\begin{enumerate}
    \item \textbf{Tree Evaluation}: Computation of raw numerical output for each training instance.
    \item \textbf{Sigmoid Activation}: Application of sigmoid function $\sigma(x) = \frac{1}{1+e^{-x}}$ to map outputs to $[0,1]$ range.
    \item \textbf{Threshold Classification}: Binary classification using 0.5 threshold.
    \item \textbf{Accuracy Calculation}: Computation of classification accuracy as the proportion of correct predictions.
\end{enumerate}

\paragraph{Parsimony Pressure}
To control tree bloat, the implementation incorporates parsimony pressure through fitness adjustment:
\begin{equation}
\text{adjusted\_fitness} = \text{accuracy} \times (0.99)^{\text{tree\_size}/50}
\end{equation}
This mechanism penalizes larger trees while preserving classification performance, encouraging the evolution of compact and interpretable solutions.

\subsubsection{Data Processing Pipeline}
\paragraph{Feature Normalization}
The DataParser class implements min-max normalization to standardize input features:
\begin{enumerate}
    \item \textbf{Range Calculation}: Determination of minimum and maximum values for each feature across the training dataset.
    \item \textbf{Normalization}: Linear scaling of each feature to $[0,1]$ range using the formula: $\frac{\text{value} - \text{min}}{\text{max} - \text{min}}$.
    \item \textbf{Constant Handling}: Assignment of 0.5 value for features with zero variance to prevent numerical instability.
\end{enumerate}

\paragraph{Data Validation}
The system ensures data integrity through:
\begin{itemize}
    \item Header row skipping during CSV parsing.
    \item Binary label enforcement (values $> 0$ mapped to 1, others to 0).
    \item Exception handling for malformed data entries.
    \item Parallel processing support for fitness evaluation.
\end{itemize}

\subsubsection{Population Management}
\paragraph{Population Structure}
The Population class maintains a collection of 1000 individuals, representing diverse mathematical expressions for the classification task. Population management includes:
\begin{itemize}
    \item \textbf{Initialization}: Random generation of initial population using ramped half-and-half method.
    \item \textbf{Evolution}: Generational replacement with complete population turnover.
    \item \textbf{Diversity Maintenance}: Stochastic selection and variation operators to preserve genetic diversity.
\end{itemize}

\paragraph{Termination Criteria}
The algorithm employs multiple termination conditions:
\begin{itemize}
    \item \textbf{Maximum Generations}: Hard limit of 100 generations to prevent infinite execution.
    \item \textbf{Early Stopping}: Validation-based stopping with patience parameter of 5 generations.
    \item \textbf{Convergence Detection}: Monitoring of validation accuracy improvement to identify optimal stopping points.
\end{itemize}

\subsubsection{Performance Optimization Features}
\paragraph{Parallel Processing}
The implementation leverages Java 8 parallel streams for fitness evaluation, enabling concurrent processing of multiple individuals during population assessment. This approach significantly reduces computational time for large populations.

\paragraph{Memory Management}
Efficient memory utilization through:
\begin{itemize}
    \item Immutable node structures to prevent unintended modifications.
    \item Proper cloning mechanisms for genetic operations.
    \item Garbage collection optimization through object lifecycle management.
\end{itemize}

\subsubsection{Validation and Testing Framework}
\paragraph{Cross-Validation Strategy}
The algorithm implements a validation split approach:
\begin{itemize}
    \item \textbf{Training Set}: 80\% of data used for fitness evaluation and evolution.
    \item \textbf{Validation Set}: 20\% of training data reserved for early stopping decisions.
    \item \textbf{Test Set}: Independent dataset for final performance evaluation.
\end{itemize}

\paragraph{Performance Metrics}
Comprehensive evaluation includes:
\begin{itemize}
    \item \textbf{Accuracy}: Overall classification correctness.
    \item \textbf{F1-Score}: Harmonic mean of precision and recall for balanced evaluation.
    \item \textbf{Generalization Assessment}: Comparison between training and test performance.
\end{itemize}

\subsubsection{Implementation Robustness}
\paragraph{Numerical Stability}
The SafeDivideNode implementation prevents division by zero through threshold checking ($|\text{denominator}| < 1 \times 10^{-5}$), returning zero for undefined operations. This approach maintains program execution stability while preserving mathematical meaning.

\paragraph{Reproducibility}
Seed-based random number generation ensures experimental reproducibility. The system accepts seed values as command-line arguments and propagates them throughout all stochastic components.

\subsubsection{Algorithm Parameters}
The key parameters used in the GP implementation are summarized in Table~\ref{tab:gp_parameters}.
\begin{table}[htbp]
\centering
\caption{Genetic Programming Algorithm Parameters}
\label{tab:gp_parameters}
\begin{tabular}{|l|c|}
\hline
\textbf{Parameter} & \textbf{Value} \\
\hline
Population Size & 1000 \\
Maximum Generations & 100 \\
Maximum Tree Depth & 4 \\
Maximum Node Count & 100 \\
Mutation Rate & 15\% \\
Tournament Size & 5 \\
Early Stopping Patience & 5 generations \\
Parsimony Pressure Factor & 0.99 \\
Division Safety Threshold & $1 \times 10^{-5}$ \\
\hline
\end{tabular}
\end{table}

\subsubsection{Conclusion}
The implemented Genetic Programming classifier demonstrates a comprehensive approach to evolutionary computation for financial classification tasks. The design incorporates established GP principles while addressing practical considerations such as numerical stability, computational efficiency, and result reproducibility. The modular architecture supports extensibility and maintenance while providing robust performance for binary classification scenarios.

\subsection{Multi-Layer-Perceptron Algorithm}
The MLP uses five inputs from the dataset: 'Open', 'High', 'Low', 'Close', 'Adj Close'. The 'Output' column from the dataset is used to verify the predicted classification against to determine the model's accuracy.

The MLP is structured as into the following layers:
\begin{itemize}
\item Inputs: 5 price features ('Open', 'High', 'Low', 'Close', 'Adj Close')
\item Hidden Layer 1: 128 neurons
\item Hidden Layer 2: 64 neurons 
\item Hidden Layer 3: 32 neurons
\item Output Layer: 1 neuron with sigmoid activation
\end{itemize}

\subsubsection{Regularization Techniques}
To prevent overfitting and improve generalization, several regularization methods were implemented:
\begin{itemize}
\item \textbf{Dropout}: 30\% of neurons were randomly deactivated during training
\item \textbf{Normalization}: Data is normalized at each hidden layer
\item \textbf{Early Stopping}: Training halted when validation loss stopped improving for 15 epochs
\end{itemize}

\subsubsection{Training Configuration}
The model was trained with the following hyperparameters:
\begin{itemize}
\item \textbf{Loss Function}: Binary cross-entropy for classification
\item \textbf{Optimizer}: Adam with learning rate of 0.001
\item \textbf{Batch}: 32 samples per batch
\item \textbf{Max Epochs}: 100
\item \textbf{Validation Split}: 20\% of training data was used for the validation of its performance
\item \textbf{Class Weights}: Applied to handle imbalanced buy/sell signals
\end{itemize}

Initially all weight values between neurons are random. During forward propagation, input features pass through each layer, with ReLU activation functions introducing non-linearity in hidden layers. After the initial run we perform the loss calculation which compares the result at the output layer to the actual value in the data using binary cross-entropy loss. Backpropagation is used to determine how each weight value contributed to the error. This information is then used to adjust the weight values to help the algorithm "learn" from previous runs. The updating of the weights is done by Adam optimizer which adaptively adjusts the learning rate for each parameter using the following simplified formula:

\begin{equation}
{weight_{new}} =  weight_{old} - (learning_{rate} \times gradient \times adam_{adjustments})
\end{equation}

where $adam_{adjustments}$ uses momentum and adaptive learning rate scaling based on historical gradients.

Sigmoid activation is used at the output neuron to classify the output as binary, producing a probability between 0 and 1. Where a value greater than or equal to 0.5 results in a 1 (buy signal) and a value less than 0.5 results in a 0 (sell signal).

\subsubsection{Model Persistence}
To ensure reproducible results, the trained model was saved after training completion. Subsequent evaluations used the saved model rather than retraining, eliminating variability from random weight initialization. A fixed random seed of 42 was used for consistent results across multiple runs.

\subsection{Decision Tree Algorithm}
The three main constraints used in the ACO algorithm are:
\begin{itemize}
\item Each vehicles route must not exceed tim/distance limit ($t_{max}$)
\end{itemize}
To solve this we simply have to check what a vehicles current distance/time value is before select a next node and then check whether this value plus the time taken to reach the candidate node and get back to the end node is less than its ($t_{max}$) value. As shown in algorithm (1).

\begin{itemize}
\item Each node can be visited visited by only 1 vehicle
\end{itemize}
For this constraint, we maintain a list of nodes ($unvitednodes$) which stores all the nodes that have not yet been visited. After the construction of each route we update this list by removing the nodes that were used in the vehicles route construction.

\begin{itemize}
\item All vehicles must begin at the indicated start node and finish at the end node.
\end{itemize}
All vehicles have a ($path$) vector attribute which is initialized to contain the value 0 at the start. The route is then constructed from this point until, there are no viable option for the node to move to. When this is the case, the node is forced to return to the end node. It is guaranteed to reach the end node due to the check made in algorithm (1).

\section{Results}

% Summary table of all algorithms
\begin{table}[h]
\centering
\begin{tabular}{|l|c|c|c|c|c|}
\hline
{\textbf{Model}} & {\textbf{Seed value}} & \multicolumn{2}{c|}{\textbf{Training}} & \multicolumn{2}{c|}{\textbf{Testing}} \\
\cline{3-6}
& & \textbf{Acc} & \textbf{F1} & \textbf{Acc} & \textbf{F1} \\
\hline
Genetic Programming & & & & & \\
\hline
MLP & 42 & 0.7244 & 0.7921 & 0.9468 & 0.9449\\
\hline
Decision Tree & & & & & \\
\hline
\end{tabular}
\caption{Performance comparison of machine learning algorithms}
\label{tab:results}
\end{table}

\clearpage
\section{Analysis}

Several key observations can be made from the results of the ACO above.

\subsection{Problem Size Affect}

The ACO algorithm seemed to find and maintain solutions to smaller problems such as those with 33 nodes quickly but failed to improve on them. Perhaps more exploration may be needed.

\subsection{Impact of Vehicle Count and Time Constraints}

\begin{itemize}
    \item When more vehicles are available (e.g., p3.4.s.txt with 4 vehicles), the algorithm achieves higher scores compared to instances with fewer vehicles (e.g., p3.2.a.txt with 2 vehicles).
    \item Increasing ($t_{max}$) leads to higher scores, as vehicles can visit more high score nodes. Compare p3.2.a.txt ($t_{max}=7.5$, score=90) with p3.2.c.txt ($t_{max}=12.5$, score=160).
\end{itemize}

\subsection{Potential Improvements}

To address the limited improvement observed in some instances, several modifications could be made:

\begin{itemize}
    \item \textbf{Dynamic Parameter Adjustment}: Adapting parameters during the run based on convergence behavior could help escape local optima.
    
    \item \textbf{Min-Max Pheromone Limits}: Implementing upper and lower bounds on pheromone values might prevent premature convergence.
    
    \item \textbf{Solution Diversity Mechanisms}: Introducing mechanisms to maintain diversity in the population, such as different pheromone matrices for subsets of ants or occasional random restarts.
    
    \item \textbf{Local Search}: Incorporating local search procedures to refine solutions could improve the quality of results, especially when the ACO reaches a plateau.
\end{itemize}

Overall, the ACO algorithm has demonstrated it was effective in solving the TOP across various datasets, particularly in larger, more complex instances.

\end{document}